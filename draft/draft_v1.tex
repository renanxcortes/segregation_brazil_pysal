\documentclass[12pt,a4paper]{article}

% -------------------------------------------------
% PACKAGES
% -------------------------------------------------
\usepackage[utf8]{inputenc}
\usepackage[T1]{fontenc}
\usepackage{lmodern}
\usepackage{geometry}
\geometry{margin=2.5cm}
\usepackage{graphicx}
\usepackage{booktabs}
\usepackage{amsmath, amssymb}
\usepackage{caption}
\usepackage{subcaption}
\usepackage{hyperref}
\usepackage{setspace}
\usepackage{float}

% Bibliography with biblatex (recommended)
\usepackage[backend=biber,style=authoryear]{biblatex}
\addbibresource{references.bib}

% -------------------------------------------------
% TITLE INFORMATION
% -------------------------------------------------
\title{\textbf{Title of the Scientific Paper}}

\author{
    First Author\thanks{Corresponding author: email@example.com} \\
    \small Department, University, Country
    \and
    Second Author \\
    \small Department, University, Country
}

\date{\today}

% -------------------------------------------------
% DOCUMENT
% -------------------------------------------------
\usepackage{Sweave}
\begin{document}
\Sconcordance{concordance:draft_v1.tex:draft_v1.Rnw:1 41 1 1 0 94 1}


\maketitle

\begin{abstract}
This is the abstract of the scientific paper. It should briefly present the objective, methodology, main results, and conclusions. Citations can also appear here if necessary \parencite{smith2020}.
\end{abstract}

\noindent\textbf{Keywords:} keyword1; keyword2; keyword3

\onehalfspacing

% -------------------------------------------------
\section{Introduction}

Introduce the research problem and provide background. 
Cite references as needed \parencite{doe2019}.

\subsection{Motivation}

Explain why the study is relevant.

\subsection{Objectives}

Clearly state the objectives of the paper.

% -------------------------------------------------
\section{Materials and Methods}

Describe the methodology used.

\subsection{Study Design}

Provide details of experimental or analytical design.

\subsection{Statistical Analysis}

Include equations if necessary:

\begin{equation}
Y_i = \beta_0 + \beta_1 X_i + \varepsilon_i
\end{equation}

% -------------------------------------------------
\section{Results}

Present findings clearly.

\subsection{Descriptive Statistics}

Example of a professional table:

\begin{table}[H]
\centering
\caption{Descriptive statistics of the sample.}
\begin{tabular}{lccc}
\toprule
Variable & Mean & SD & N \\
\midrule
Variable 1 & 10.5 & 2.3 & 100 \\
Variable 2 & 5.7  & 1.8 & 100 \\
\bottomrule
\end{tabular}
\end{table}

\subsection{Figures}

Example of figure insertion:

\begin{figure}[H]
\centering
\includegraphics[width=0.7\textwidth]{grid_plot_4314100_systematic.png}
\caption{Example of a figure.}
\end{figure}

% -------------------------------------------------
\section{Discussion}

Interpret the results and compare with literature \parencite{smith2020}.

% -------------------------------------------------
\section{Conclusion}

Summarize main findings and implications.

% -------------------------------------------------
\section*{Acknowledgments}

Optional acknowledgment section.

% -------------------------------------------------
\printbibliography

\end{document}
